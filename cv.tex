%-------------------------
% Jacob J. Kitzmann Resume in Latex
% Based off of: https://github.com/jakeryang/resume
%------------------------

\documentclass[letterpaper,11pt]{article}

\usepackage{latexsym}
\usepackage[empty]{fullpage}
\usepackage{titlesec}
\usepackage{marvosym}
\usepackage[usenames,dvipsnames]{color}
\usepackage{verbatim}
\usepackage{enumitem}
\usepackage[hidelinks]{hyperref}
\usepackage{fancyhdr}
\usepackage[english]{babel}
\usepackage{tabularx}

% fontawesome
\usepackage{fontawesome5}

% fixed width
\usepackage[scale=0.90,lf]{FiraMono}

% light-grey
\definecolor{light-grey}{gray}{0.83}
\definecolor{dark-grey}{gray}{0.3}
\definecolor{text-grey}{gray}{.08}

\DeclareRobustCommand{\ebseries}{\fontseries{eb}\selectfont}
\DeclareTextFontCommand{\texteb}{\ebseries}

% custom underilne
\usepackage{contour}
\usepackage[normalem]{ulem}
\renewcommand{\ULdepth}{1.8pt}
\contourlength{0.8pt}
\newcommand{\myuline}[1]{%
  \uline{\phantom{#1}}%
  \llap{\contour{white}{#1}}%
}


% custom font: helvetica-style
\usepackage{tgheros}
\renewcommand*\familydefault{\sfdefault} 
%% Only if the base font of the document is to be sans serif
\usepackage[T1]{fontenc}


\pagestyle{fancy}
\fancyhf{} % clear all header and footer fields
\fancyfoot{}
\renewcommand{\headrulewidth}{0pt}
\renewcommand{\footrulewidth}{0pt}

% Adjust margins
\addtolength{\oddsidemargin}{-0.5in}
\addtolength{\evensidemargin}{0in}
\addtolength{\textwidth}{1in}
\addtolength{\topmargin}{-.5in}
\addtolength{\textheight}{1.0in}

\urlstyle{same}

\raggedbottom
\raggedright
\setlength{\tabcolsep}{0in}

\titleformat {\section}{
    \bfseries \vspace{2pt} \raggedright \large % header section
}{}{0em}{}[\color{light-grey} {\titlerule[2pt]} \vspace{-4pt}]


\newcommand{\resumeItem}[1]{
  \item\small{{#1 \vspace{-1pt}}}}

\newcommand{\resumeSubheading}[4]{
  \vspace{-1pt}\item
    \begin{tabular*}{\textwidth}[t]{l@{\extracolsep{\fill}}r}
      \textbf{#1} & {\color{dark-grey}\small #2}\vspace{1pt}\\ % top row of resume entry
      \textit{#3} & {\color{dark-grey} \small #4}\\ % second row of resume entry
    \end{tabular*}\vspace{-4pt}
}

\newcommand{\resumeSubSubheading}[2]{
    \item
    \begin{tabular*}{\textwidth}{l@{\extracolsep{\fill}}r}
      \textit{\small#1} & \textit{\small #2} \\
    \end{tabular*}\vspace{-7pt}
}


\newcommand{\resumeSubItem}[1]{\resumeItem{#1}\vspace{-4pt}}

\renewcommand\labelitemii{$\vcenter{\hbox{\tiny$\bullet$}}$}

% CHANGED default leftmargin  0.15 in
\newcommand{\resumeSubHeadingListStart}{\begin{itemize}[leftmargin=0in, label={}]}
\newcommand{\resumeSubHeadingListEnd}{\end{itemize}}
\newcommand{\resumeItemListStart}{\begin{itemize}}
\newcommand{\resumeItemListEnd}{\end{itemize}\vspace{0pt}}

\color{text-grey}

\begin{document}

\begin{center}
    \textbf{\Huge Jacob Jeffery Kitzmann} \\ \vspace{5pt}
    \small \texttt{614 E. Jefferson St Iowa City, IA 52245} \hspace{1pt} $|$
    \hspace{1pt} \hspace{2pt} \texttt{jacobkitz@gmail.com} \hspace{1pt} 
    \\ \vspace{-3pt}
\end{center}

\section {RESEARCH INTEREST}
Medical image processing and analysis -- Photon Counting Computed Tomography -- Acquisition/reconstruction protocol development -- Pulmonary imaging

\section {EDUCATION}
    \resumeSubHeadingListStart
    \resumeSubheading
        {Doctor of Philosophy in Biomedical Engineering}
        {June 2025 -- Present}
        {The University of Iowa}
        {Iowa City, IA}
            \resumeItemListStart
            \resumeItem {Certificate in College Teaching}
            \resumeItem {Triathlon Club Team}
            \resumeItemListEnd
    \resumeSubheading
        {Bachelor of Science in Biomedical Engineering}
        {August 2020 -- May 2025}
        {The University of Iowa}
        {Iowa City, IA}
          	\resumeItemListStart
            \resumeItem {Minor in Mathematics}
        	\resumeItem {Ski and Snowboard club, Pi Kappa Phi Fraternity}
            \resumeItemListEnd
    \resumeSubHeadingListEnd

\section {RESEARCH EXPERIENCE}
    \resumeSubHeadingListStart
    \resumeSubheading
        {Graduate Research Student}
        {June 2025 -- Present}
        {University of Iowa Health Care Dept. of Radiology}
        {Iowa City, IA}
            \resumeItemListStart
            \resumeItem {Developed novel image analysis pipeline for partial volume averaging quantification between voxel dimensions}
            \resumeItem {}
            \resumeItemListEnd

    \resumeSubheading
        {Undergraduate Research Student}
        {October 2022 -- May 2025}
        {University of Iowa Health Care Dept. of Radiology}
        {Iowa City, IA}
            \resumeItemListStart
            \resumeItem {Created manual segmentations of pulmonary nodules for use in ensemble-based neural network malignancy risk prediction system.}
            \resumeItem {Visualized quantitative CT data for acquisition and reconstruction protocol development in photon counting computed tomography}
            \resumeItem {Drafted manuscripts for journal articles and conference abstract submissions}
            \resumeItemListEnd
    \resumeSubHeadingListEnd

\section {TEACHING EXPERIENCE}
    \resumeSubHeadingListStart
    \resumeSubheading
        {Biomedical Engineering Internship Mentor}
        {June 2025 - August 2025}
        {Kirkwood CC Workplace Learning Connection}
        {Iowa City, Iowa}
        \resumeItemListStart
        \resumeItem {Introduced a high school student to the field of biomedical engineering through medical equipment restoration of: ventilators, electrocardiograms, bronchoscopes}
        \resumeItem {Taught student basic circuit design and electrical safety, resulting in full restoration of a non-functional Harvard Apparatus ventilator}
        \resumeItemListEnd
    \resumeSubheading
        {Teaching Assistant}
        {January 2025 -- May 2025}
        {ENGR:3110 -- Intro to AI and Machine Learning in Engineering}
        {}
        \resumeItemListStart
        \resumeItem {Developed and delivered lectures on image processing fundamentals, computer vision, and deep learning.}
        \resumeItem {Created syllabus for course final project on image classification and assisted students with Python and Jupyter Notebook implementation.}
        \resumeItem {Taught students to create and utilize reproducible science environments through Anaconda, PyTorch, Scikit-Learn}
        \resumeItemListEnd
    \resumeSubheading
        {Teaching Assistant}
        {August 2024 - December 2024}
        {BME:2400 -- Cell Biology for Engineers}
        {}
        \resumeItemListStart
        \resumeItem {Graded homework, lab reports, and exams}
        \resumeItem {Hosted office hours for student assistance in course material and administrative details.}
        \resumeItemListEnd
    \resumeSubheading 
        {Near-Peer Mentor}
        {June 2024 - August 2024}
        {Cancer Research Opportunities at Iowa}
        {Iowa City, Iowa}
        \resumeItemListStart
        \resumeItem {Guided a summer research intern through statistical analysis in Python for clinical CT data, resulting in several presentations from the student and expansion of our research cohort}
    \resumeSubHeadingListEnd

\section {PUBLICATIONS}
    \resumeSubHeadingListStart
        \resumeSubheading
        {Journal Articles}
        {}
        {}
        {}
        \begin {enumerate}
            \item Sieren, J. C., Schroeder K., \textbf{Kitzmann, J.}, Knoernschild, K., Atha, J., Alarab, N., Guo, J., Fain, S. B., Hoffman, E. A. (2024, December). Ultra-high Resolution Photon Counting Detector Computed Tomography Imaging for Quantitative Lung Assessment: An Anthropomorphic Phantom Study. In Investigative Radiology. LWW.
        \end {enumerate}

        \resumeSubheading
        {Conference Papers}
        {}
        {}
        {}
        \begin {enumerate}
        \item \textbf{Kitzmann, J.}, Atha, J., Schroeder, K., Knoernschild, K., Hoffman, E. A., Sieren, J. C. (2026, May) Development of a Standard Field of View for Isotropic Voxel Quantitative Lung Imaging Utilizing Ultra High Resolution Photon Counting Computed Tomography. Accepted for presentation at ATS 2026 International Conference. American Thoracic Society.

        \item Knoernschild K., Schroeder K., \textbf{Kitzmann, J.}, Sieren, J. C. (2026, May) Early Prediction of Rapid Emphysema Progression in a Lung Cancer Screening Cohort Using Quantitative LDCT and an Ensemble-Based Machine Learning Model. Accepted for presentation at ATS 2026 International Conference. American Thoracic Society.

        \item Knoernschild, K., Schroeder, K. E., \textbf{Kitzmann, J.}, Colby, C., Sieren, J. C. (2025, April). Ensemble artificial neural network lung nodule classification utilizing nodular and peri-nodular radiomics. In Medical Imaging 2025: Computer-Aided Diagnosis (Vol. 13407, pp. 628-633). SPIE.
        
        \item Knoernschild, K., Schroeder, K., \textbf{Kitzmann, J.}, Colby, C., Sieren, J. C. (2025, April). Towards Improved Lung Cancer Screening: Evaluating Nodule Misclassification Trends of a Radiomics-based Prediction Model. In American Journal of Respiratory and Critical Care Medicine (Vol. 211). American Thoracic Society.
        \end {enumerate}
    
\section {PRESENTATIONS}
\begin {enumerate}

\item \textbf{Jacob J. Kitzmann}, Kimberly E. Schroeder, Jarron Atha, Kelly Stark, Sean B. Fain, Eric A. Hoffman, Jessica C. Sieren, Evaluation of Quantitative Iterative Reconstruction Kernel in Photon Counting Computed Tomography Between Animal and Phantom Studies. University of Iowa Fall Undergraduate Research Festival. October 2024

\item \textbf{Jacob J. Kitzmann}, Kimberly E. Schroeder, Kevin S. Knoernschild, Chandra E. Colby, and Jessica C. Sieren. Radiomic Quantification from Low Dose Chest CT for Prediction of COPD Progression. University of Iowa College of Engineering Research Open House. April 2024

\item \textbf{Jacob J. Kitzmann}, Chandra E. Colby, Kevin S. Knoernschild, Kimberly E. Schroeder, and Jessica C. Sieren. Semi-automated Segmentation of Pulmonary Nodules. University of Iowa Lung Imaging Research Symposium. August 2023

\item \textbf{Jacob J. Kitzmann}, Chandra E. Colby, Kevin S. Knoernschild, Kimberly E. Schroeder, and Jessica C. Sieren. Semi-automated Segmentation of Pulmonary Nodules. University of Iowa College of Engineering Research Open House. April 2023

\end {enumerate}

\section {SCHOLARSHIPS AND AWARDS}
\begin{enumerate}
    \item \textbf{Office of Undergraduate Research Fellowship} --- \$2{,}500 \\
    Fall 2024 -- Spring 2025 \;|\; Competitive Institutional Award

    \item \textbf{Clarence H. Clark Engineering Scholarship} --- \$1{,}000 \\
    Fall 2024 -- Spring 2025 \;|\; Competitive Institutional Award

    \item \textbf{Forrest and Ada Kehn Engineering Scholarship} --- \$1{,}500 \\
    Fall 2023 -- Spring 2024 \;|\; Competitive Institutional Award

    \item \textbf{University of Iowa Dean's List} \\
    Spring 2022, Fall 2022, Fall 2023, Fall 2024, Spring 2025, Fall 2025
\end{enumerate}

\end{document}
